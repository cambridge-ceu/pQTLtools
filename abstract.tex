The difficulty with genetic study of complex traits has raised concerns over
optimal study design and statistical analysis.  The key statistical issues in
designing different studies are validity, power and robustness of relevant
statistical tests.  This thesis investigates several scenarios of power
analysis in linkage and association analysis, which include linkage tests in
small pedigrees, association tests for case-control data, marker polymorphism
and mutation detection, computer simulation methods and application.  It also
gives results of numerical experiments on family haplotype analysis and
discusses TDT and other association designs.  This thesis reveals that commonly
used parametric and nonparametric linkage statistics are comparable in power
for two point analysis with simple families, but nonparametric linkage
statistics are anticonservative in some families.  As for case-control data,
heterogeneity statistic nearly has power close to the true model without the
needs of disease model specification, and comparable to the ordinary likelihood
ratio test for contingency table.  In mutation detection, multiallelic marker
is usually more favourable than SNP.  While haplotype analysis has claimed
power for linkage and association, there may be numerical and analytical
difficulties when a likelihood approach using family data is adopted.  Finally,
correct sample sizes are obtained for TDT design as reported earlier in the
literature, and some computer routines performing these calculations are also
given.
