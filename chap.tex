\chapter{Computer programs}\label{programs}

This appendix includes some subroutines and programs used in previous chapters.

\section{C subroutines}
Subroutine {\bf fibonacci} is used to locate the interval containing optimal
value in front of a unidimensional optimisation routine.  It has been used in
Chapter 4.
\begin{verbatim}
float fibonacci(float a,float b,float(*f)(float),float eps)
/*
Brian D Bunday & Gerald R Garside (1987). Optimisation Methods in Pascal.
Edward Arnold, finalized 11/11/98 JH Zhao
*/
{
float x1, x2, x3, x4, f2, f4, fib0, fib1, fib2;
int n, k;
n=15;
fib0 = 1;
fib1 = 1;
for (k = 2; k <= n; k++) {
  fib2 = fib1 + fib0;
  fib0 = fib1;
  fib1 = fib2;
}
x1 = a;
x3 = b;
if (n % 2) eps = -eps;
x2 = a + ((b - a) * fib0 + eps) / fib1;
f2 = (*f)(x2);
for (k = 2; k <= n; k++) {
  x4 = x1 - x2 + x3;
  f4 = (*f)(x4);
  if (f4 < f2) {
    if (x4 < x2) x3 = x2; else x1 = x2;
    x2 = x4;
    f2 = f4;
  } else if (x4 < x2) x1 = x4; else x3 = x4;
}
asave=x1;
bsave=x3;
return x2;
}
\end{verbatim}

\noindent Subroutine {\bf fmin} is popular for unidimensional optimisation
(Forsythe et al.  1977; Press et al.  1992).  An auxiliary routine {\em sign}
is also given.
\begin{verbatim}
int fcount,maxn;
int sign(float a,float b)
/*an invent for Fortran SIGN(.) function in C*/
{
return((a>b)?1:((a==b)?0:-1));
}
float fmin(float ax, float bx, float (*f)(float), float tol)
/* This is Brent's method of univariate optimization adapted from
   Forsythe GE et al. (1977) and Press WH et al. (1992)
   JH Zhao 26/8/1998 IOP London
*/
{
int iter;
float a,b,d,fu,fv,fw,fx,p,q,r,tol1,tol2,u,v,w,x,xm;
float e=0.0,goldr=(3.0-sqrt(5.0))/2.0,maxiter=15,zeps=1.e-10;
a=ax;
b=bx;
w=v=x=bx;
fw=fv=fx=(*f)(x);
for (iter=1;iter<=maxiter;iter++) {
   xm=0.5*(a+b);
   tol2=2.0*(tol1=tol*fabs(x)+zeps);
   if (fabs(x-xm) <= (tol2-0.5*(b-a))) return x;
   if (fabs(e) > tol1) { /* fit parabola*/
      r=(x-w)*(fx-fv);
      q=(x-v)*(fx-fw);
      p=(x-v)*q-(x-w)*r;
      q=2.0*(q-r);
      if (q > 0.0) p = -p;
      q=fabs(q);
      r=e;
      e=d;
      if (fabs(p) < fabs(0.5*q*r) || p > q*(a-x) || p < q*(b-x))
      { /* parabolic interpolation step*/
         d=p/q;
         u=x+d;
         if (u-a < tol2 || b-u < tol2) d=sign(tol1,xm-x);
      } else /* golden section step*/
         d=goldr*(e=(x >= xm ? a-x : b-x));
   } else d=goldr*(e=(x >= xm ? a-x : b-x));
   u=(fabs(d) >= tol1 ? x+d : x+sign(tol1,d));
   fu=(*f)(u);
   if (fu <= fx) { /*update a,b,v,w,and x*/
      if (u >= x) a=x; else b=x;
      v = w; w = x; x = u;
      fv = fw; fw = fx; fx = fu;
   } else {
      if (u < x) a=u; else b=u;
      if (fu <= fw || w == x) {
         v=w; w=u;
         fv=fw; fw=fu;
      } else if (fu <= fv || v == x || v == w) {
        v=u; fv=fu;
      }
   }
}
/*perror("Maximum iterations exceeded in fmin");*/
return x;
}
\end{verbatim}

\noindent Subroutine {\bf digit}, due to Zhao \& Sham (1998), has been used to
sequentially list all possible patterns of mutations in Chapter 4.
\begin{verbatim}
int digit(int radix, int d[], int i)
/*digital addition*/
{
if (d[i] < radix) {
   ++d[i];
   goto ok;
   }
else {
   d[i] = 1;
   ++d[i + 1];
   if (d[i + 1] <= radix) goto ok;
   }
digit(radix, d, i+1);
ok:
return 0;
}
\end{verbatim}

\noindent The binary search tree routines below are used in {\bf
fastEHPLUS}.
\begin{verbatim}
typedef struct node_type
{
  int key,n;
  struct node_type *left, *right;
} node;
node *itree(node*,int),*stree(node*,int),*dtree(node*,int);
void inorder(node*),preorder(node*),postorder(node*);
void rtree(node*),ptree(node*,int),etree(node*);
node *rt=NULL;
main()
{
int s[]={4,2,1,3,6,5,7,4,2,1,3,6,5,7},i;
for(i=0;i<14;i++) rt=itree(rt,s[i]);
printf("tree: root=%d\n\n",rt->key);
ptree(rt,0);
printf("\nPreorder:");
preorder(rt);
printf("\nInorder:");
inorder(rt);
printf("\nPostorder:");
postorder(rt);
etree(rt);
ptree(rt,0);
}
node *itree(node *r,int key)
/*insert and sort*/
{
  if (r==NULL)
  {
    r=(node*)malloc(sizeof(node));
    if (r==NULL)
    {
      printf("out of memory\n");exit(0);
    }
    r->left=r->right=NULL;
    r->key=key;
    r->n=1;
  }
  else
  if (key<r->key) r->left=itree(r->left,key);
  else if (key>r->key) r->right=itree(r->right,key);
  else r->n++;
  return r;
}
node *stree(node *t,int key)
/*search*/
{
  if (!t) return t;
  while (t->key!=key)
  {
    if (key<t->key) t=t->left;
    else t=t->right;
    if (t==0) break;
  }
  return t;
}
node *dtree(node *t,int key)
/*delete*/
{
  node *p,*p2;

  if (!t) return t;
  if (t->key==key)
  {
    if(t->left==t->right)
    {
      free(t);
      return NULL;
    }
    else if (t->left==NULL)
    {
      p=t->right;
      free(t);
      return p;
    }
    else if (t->right==NULL)
    {
      p=t->left;
      free(t);
      return p;
    }
    else
    {
      p2=t->right;
      p=t->right;
      while (p->left) p=p->left;
      p->left=t->left;
      free(t);
      return p2;
    }
  }
  if (t->key<key) t->right=dtree(t->right,key);
  else t->left=dtree(t->left,key);
  return t;
}
void inorder(node *t)
/*left subtree->t->right subtree*/
{
  if (!t) return;
  inorder(t->left);
  printf("%d ",t->key);
  inorder(t->right);
}
void preorder(node *t)
/*t->left subtree->right subtree->t*/
{
  if (!t) return;
  printf("%d ",t->key);
  preorder(t->left);
  preorder(t->right);
}
void postorder(node *t)
/*left subtree->right subtree->t*/
{
  if (!t) return;
  postorder(t->left);
  postorder(t->right);
  printf("%d ",t->key);
}
void etree(node *t)
{
  printf("\nfree tree\n");
  rtree(t);
  rt=NULL;
}
void rtree(node *t)
/*left subtree->right subtree->t*/
{
  if (!t) return;
  rtree(t->left);
  rtree(t->right);
  printf("freeing %d\n",t->key);
  free(t);
}
void ptree(node *r,int l)
/*print tree inorder*/
{
  int i;
  if (!r) return;
  ptree(r->left,l+1);
  for (i=0;i<l;++i) printf("  ");
  printf("%d [%d]\n",r->key,r->n);
  ptree(r->right,l+1);
}
\end{verbatim}

\noindent Subroutines {\bf fbsize} and {\bf pbsize} are used to obtain power of
TDT and case-control designs in Chapters 4 and 7.
\begin{verbatim}
#include <stdio.h>
#include <math.h>
#define ERROR /*Science 273: 1516-17 13SEP1996*/
#undef ERROR  /*Science 275: 1327-30 28FEB1997*/
typedef struct normal{float m,v;} norm;
int fbsize(float,float),pbsize();
float n(norm,int);
float y,h,h1,h2,p_A,lambda_o,lambda_s,n1,n2,n3;
float models[12][2]={
    4.0, 0.01,      4.0, 0.10,      4.0, 0.50,      4.0, 0.80,
    2.0, 0.01,      2.0, 0.10,      2.0, 0.50,      2.0, 0.80,
    1.5, 0.01,      1.5, 0.10,      1.5, 0.50,      1.5, 0.80};
main()
{
float gamma,p;
int i;
/*Risch & Merikangas 1996*/
printf("\nThe family-based result: \n");
printf("\ngamma   p      Y     N_asp    P_A    Het
   N_tdt   Het N_asp/tdt  L_o   L_s\n\n");
for(i=0;i<12;i++) {
  gamma=models[i][0];
  p=models[i][1];
  fbsize(gamma,p);
  if((i+1)%4==0) printf("\n");
}
/*APOE-4, Scott WK, Pericak-Vance, MA & Haines JL*/
/*Genetic analysis of complex diseases 1327*/
gamma=4.5;
p=0.15;
printf("Alzheimer's:\n\n");
fbsize(gamma,p);
/*Long AD, Grote MN & Langley CH 1328*/
pbsize();
}

int fbsize(float gamma,float p)
/*Jing Hua Zhao 30-12-98*/
{
norm nl, al, na, aa;
float k,va,vd,q,w;
q=1-p;
k=pow(p*gamma+q,2);
va=2*p*q*pow((gamma-1)*(p*gamma+q),2);
vd=pow(p*q*pow(gamma-1,2),2);
#ifdef DEBUG
printf("K=%f VA=%f VD=%f\n",k,va,vd);
#endif
w=p*q*pow(gamma-1,2)/pow(p*gamma+q,2);
y=(1+w)/(2+w);
lambda_s=pow(1+0.5*w,2);
lambda_o=1+w;
h=h1=p*q*(gamma+1)/(p*gamma+q);
p_A=gamma/(gamma+1);
/*ASP*/
nl.m=0;
nl.v=1;
al.m=2*y-1;
#ifdef ERROR
al.v=0;
#else
al.v=4*y*(1-y);
#endif
n1=n(al,1);
/*TDT*/
na.m=0;
na.v=1;
aa.m=sqrt(h)*(gamma-1)/(gamma+1);
aa.v=1-h*pow((gamma-1)/(gamma+1),2);
n2=n(aa,2);
/*ASP-TDT*/
h=h2=p*q*pow(gamma+1,2)/(2*pow(p*gamma+q,2)+p*q*pow(gamma-1,2));
aa.m=sqrt(h)*(gamma-1)/(gamma+1);
aa.v=1-h*pow((gamma-1)/(gamma+1),2);
n3=n(aa,3);
printf("%4.1f%5.2f|%6.3f%10.0f|%6.3f|%6.3f%8.0f%6.3f%8.0f|%6.2f%6.2f\n",
       gamma,p,y,n1,p_A,h1,n2,h2,n3,lambda_o,lambda_s);
return 0;
}

float n(norm all,int op)
/*m=0,v=1 under the null hypotheses*/
{
float z_alpha,z1_beta;
float s,m,v;
m=all.m;
v=all.v;
z1_beta=-0.84;                       /*1-beta=0.8,-.84162123*/
switch(op) {
case 1:z_alpha=3.72;break;           /*alpha=1E-4,3.7190165*/
default:z_alpha=5.33;break;          /*alpha=5E-8,5.3267239*/
}
s=pow((z_alpha-sqrt(v)*z1_beta)/m,2)/2; /*shared/transmitted for each parent*/
if(op==3) s/=2;                         /*the sample size is halved*/
return (s);
}

#define x2_alpha 29.72 /*Q=29.7168*/
#define z_alpha 5.45   /*5.4513104*/

int pbsize()
{
int i,j;
float z1_beta,gamma,lambda,pi,p,q,n,k,s,kp[3]={0.01,0.05,0.10};
/*
\alpha=5e-8,\beta=0.8, \alpha would give 5\% genome-wide significance level
\lambda is the NCP from the marginal table
\pi is the pr(Affected|aa)
*/
gamma=4.5; /*again for Alzheimer's*/
p=0.15;
q=1-p;
pi=0.065;  /*0.07 generates 163, equivalent to ASP*/
z1_beta=-0.84;
k=pi*pow(gamma*p+q,2);
s=(1-pi*pow(gamma,2))*pow(p,2)+(1-pi*gamma)*2*p*q+(1-pi)*pow(q,2);
/*LGL formula*/
lambda=pi*(pow(gamma,2)*p+q-pow(gamma*p+q,2))/(1-pi*pow(gamma*p+q,2));
/*my own*/
lambda=pi*p*q*pow(gamma-1,2)/(1-pi*pow(gamma*p+q,2));
/*not sure about +/-!*/
n=pow(z1_beta+z_alpha,2)/lambda;
/*may be used to correct for population prevalence*/
printf("\nThe population-based result: Kp=%f Kq=%f n=%10.0f\n",k,s,n);
printf("\nRandom ascertainment with disease prevalence\n");
printf("\n         1%%         5%%        10%%\n\n");
for(i=0;i<12;i++) {
  gamma=models[i][0];
  p=models[i][1];
  q=1-p;
  for(j=0;j<3;j++) {
    k=kp[j];
    pi=k/pow(gamma*p+q,2);
    lambda=pi*p*q*pow(gamma-1,2)/(1-pi*pow(gamma*p+q,2));
    n=pow(z1_beta-z_alpha,2)/lambda;
    printf(" %10.f",n);
  } printf("\n");
  if((i+1)%4==0) printf("\n");
}
printf("This is only an approximation, a more accurate result\n");
printf("can be obtained by Fisher's exact test\n");
return 0;
}
\end{verbatim}

\section{SAS programs}

\noindent Program {\bf power} obtains noncentrality parameter of $\chi_f^2$
distribution, $f=1, \ldots, 30$, for a given power and significant level
$\alpha$.
\begin{verbatim}
title;
data power;
do f=1 to 30;
   alpha=0.05;
   power=0.8;
   c=cinv(1-alpha,f);
   NCP=cnonct(c,f,1-power);
   p=probchi(c,f,c);
   d=cinv(1-alpha,f,0.87);
   output;
end;
proc print;run;
\end{verbatim}

\noindent Program {\bf ncp} includes a macro to obtain the required sample
sizes using means of log-likelihood ratio tests.
\begin{verbatim}
/*Adapted from ncp.sas (use macro) JH Zhao 1-8-2002, redo 8-8-2002*/
options nocenter nonumber;
%macro n(data=abc);
data abc;
  set &data;
  array m{7} T H F R D P G;
  array s{7} st sh sf sr sd sp sg;
  array nn{7} nnt nnh nnf nnr nnd nnp nng;
  do i=1 to 7;
     if i=3 then nn[i]=ceil(18.3/(m[i]/nobs));
     else nn[i]=ceil(17.4/(m[i]/nobs));
  end;
  array n{7} nt nh nf nr nd np ng;
  array k{7} kt kh kf kr kd kp kg;
  do i=1 to 7;
     if (i=3) then do;
        n[i]=ceil(18.3/(m[i]-7)*nobs);
        k[i]=ceil(18.3/((s[i]*s[i]-2*7)/4)*nobs);
     end;
     else do;
        n[i]=ceil(17.4/(m[i]-6)*nobs);
        k[i]=ceil(17.4/((s[i]*s[i]-2*6)/4)*nobs);
     end;
  end;
  drop i;
run;
proc print noobs;
     var model nnt nnr nnd nnf nnh nnp nng;
     where model ne 'NULL';
run;
proc print noobs;
     var model nt nr nd nf nh np ng;
     where model ne 'NULL';
run;
proc print noobs;
     var model kt kr kd kf kh kp kg;
     where model ne 'NULL';
run;
proc print noobs;
     var model t r d f h p g st sr sd sf sh sp sg;
     format _numeric_ 7.2;
run;
%mend n;
data abc10000;
  nobs=10000;
  do model='NULL','RR','RD','CR','CD','ALZ';
     input name$ T H F R D P G/name$ st sh sf sr sd sp sg;
     output;
  end;
cards;
mean     6.086     6.161     6.244     6.153     6.159     6.144     6.157
 std     3.481     3.552     3.595     3.550     3.557     3.533     3.548
mean  3199.286  3199.277  3199.285  3199.285  2356.602  3063.708  3197.999
 std   111.160   111.121   111.105   111.105    77.514   104.428   110.949
mean  1041.946   907.501  1041.946   907.501  1041.946   895.769   907.543
 std    57.352    52.210    57.357    52.202    57.352    50.888    52.217
mean   804.946   716.690   805.366   716.718   541.284   710.195   716.737
 std    60.346    57.130    60.390    57.128    42.141    56.103    57.127
mean   261.983   252.311   262.728   252.311   262.585   251.125   252.346
 std    31.544    30.534    31.622    30.536    31.631    30.246    30.529
mean   268.792   270.229   270.977   270.353   261.748   269.156   270.263
 std    32.161    32.490    32.454    32.487    31.324    32.233    32.499
;
data abc1000;
  nobs=1000;
  do model='NULL','RR','RD','CR','CD','ALZ';
     input name$ T H F R D P G/name$ st sh sf sr sd sp sg;
     output;
  end;
cards;
mean     5.393     6.173     6.377     6.148     6.084     6.032     6.177
 std     2.802     3.366     3.416     3.330     3.219     3.221     3.379
mean   327.218   327.218   327.295   327.218   242.027   312.990   327.172
 std    33.957    33.961    33.977    33.959    23.160    31.830    33.919
mean   109.637    96.629   109.785    96.629   109.637    95.050    96.638
 std    19.155    18.022    19.231    18.023    19.155    17.463    18.025
mean    85.210    77.086    86.309    77.089    59.308    76.035    77.082
 std    18.618    17.870    18.807    17.870    13.341    17.397    17.872
mean    30.568    30.370    32.022    30.370    31.399    30.045    30.370
 std    10.170    10.204    10.425    10.205    10.315    10.037    10.204
mean    31.881    32.896    33.842    32.881    31.944    32.560    32.901
 std    10.876    11.132    11.300    11.138    10.725    10.980    11.138
;
%n(data=abc10000);
%n(data=abc1000);
\end{verbatim}

\noindent Macro {\bf setf} obtains disease models compatible with population
disease prevalence and genotypic distribution.  An example of Alzheimer's
disease is also provided.
\begin{verbatim}
/***********************************************************/
/*This SAS macro gives a set of kp compatible q and f's.   */
/*      kp=population prevalence                           */
/*f0,f1,f2=the penetrance                                  */
/*       q=the putative disease allele frequency           */
/*Corrected 4/8/97 IOP JH ZHAO                             */
/***********************************************************/
%macro setf(setq=,setf0=,setf1=,setf2=);
title;
options nocenter nodate nonumber;
proc iml;
q=&setq;p=1-q;
f0=&setf0;f1=&setf1;f2=&setf2;
p1=p;p2=q;
k=f0*p**2+f1*2*p*q+f2*q**2;
f=(f0//f1//f2);
b=(f0*p**2/k)//(f1*2*p*q/k)//(f2*q**2/k);
print k q f b;
step1=q/10;
step2=p*q/10;
do q=0.0 to 1.0 by step1;
   p=1-q;
   do d=-p*q to p*q by step2;
      h11=p*p1+d;
      h12=p*p2-d;
      h21=q*p1-d;
      h22=q*p2+d;
      h=h11+h12+h21+h22;
      a=(  (h11*h11 || 2*h11*h21           || h21*h21)//
         (2*h11*h12 || 2*h11*h22+2*h12*h21 || 2*h21*h22)//
           (h12*h12 || 2*h12*h22           || h22*h22)
        )/k;
      f=ginv(a)*b;
      kp=f[1]*p**2+f[2]*2*p*q+f[3]*q**2;
      hh=h11//h12//h21//h22;
      c=a[+,]*k;
      if (h11>=0 && h12>=0 && h21>=0 && h22>=0) &&
         (f[1]>=0 && f[1] <= f[2] && f[2]<=f[3] && f[3]<=1)
      then print hh q d a f kp;
   end;
end;
quit;
%mend setf;
%setf(setq=0.13,setf0=0.05,setf1=0.2,setf2=0.8);
\end{verbatim}

\section{QBASIC program}
{\bf N.BAS} illustrates how mixture of $\chi^2$ is used for obtaining
required sample size and has been used in Chapter 6.
\begin{verbatim}
DECLARE SUB ssize (pd, p1, r!, g!)
PRINT
PRINT "pd", "p1", "r", "g", "n (t=10cM)"
PRINT
CALL ssize(.1, .1, 4, 15): CALL ssize(.5, .1, 4, 15)
CALL ssize(.1, .3, 4, 15): CALL ssize(.5, .3, 4, 15)
CALL ssize(.1, .5, 4, 15): CALL ssize(.5, .5, 4, 15)
CALL ssize(.1, .7, 4, 15): CALL ssize(.5, .7, 4, 15)
CALL ssize(.1, .9, 4, 15): CALL ssize(.5, .9, 4, 15)
PRINT
CALL ssize(.1, .1, 4, 50): CALL ssize(.5, .1, 4, 50)
CALL ssize(.1, .3, 4, 50): CALL ssize(.5, .3, 4, 50)
CALL ssize(.1, .5, 4, 50): CALL ssize(.5, .5, 4, 50)
CALL ssize(.1, .7, 4, 50): CALL ssize(.5, .7, 4, 50)
CALL ssize(.1, .9, 4, 50): CALL ssize(.5, .9, 4, 50)
END

' sample size required for LLD with S families
'
' pd=disease allele frequency
' pn=normal allele frequency
' p1=associated allele frequency for biallelic locus
' n=sample size
' t=theta, recombination fraction, 10cM and Haldane function
' r=relative risk assuming multiplicative model
' psi as in Haseman and Elston (1972)
' K=population prevalence
' g=generations
' s=required sample size
' critical value for 0.5*X^2(1)+0.5*X^2(2)=3.77*2*2.3025851
' x2n=noncentrality parameter for alpha=0.0001, beta=0.2
' di npnchi(2,3.77*2*2.3025851,0.2) " " nchi(2,23.978829,3.77*2*2.3025851)
' 23.978829 .1999992
'
SUB ssize (pd, p1, r, g)
t = .0906  '10cM
psi = t ^ 2 + (1 - t) ^ 2
pn = 1 - pd
k = (pd * r + pn) ^ 2
ncp = (4 * psi / k) * ((r - 1)*pd) ^ 2 * ((1 - p1) / p1) * EXP(-2 * t * g)
x2n = 23.978829#
n = x2n / ncp
PRINT pd, p1, r, g, n

END SUB
\end{verbatim}

%ms_model()
%/*Ebers G et al (1996) on multiple sclerosis Nat Genet 13:472*/
%/*for modest \lambda_s, two models are the same*/
%/*\lambda_s=sibling risk ratio associated with the locus*/
%{
%float z2,z1,z0,y;
%
%/*model with moderate amount of dominance variance*/
%y=1-0.5/sqrt(lambda_s);
%z2=y*y;
%z1=2*y*(1-y);
%z0=pow(1-y,2);
%
%/*additive model, dominance variance eq 0*/
%z2=0.5-0.25/lambda_s;
%z1=0.5;
%z0=0.25/lambda_s;
%}

%\section{Related publications}
%\noindent Chapter 2: Sham et al. (1997,2000) AJHG.
%\noindent Chapter 3: Zhao \& Sham (1997) AJMD, Zhao et al. (2000). HH
%\noindent Chapter 3: Zhao \& Sham (2001) CMPB
%\noindent Chapter 4: Zhao \& Sham (1998) CSDA. Sham et al. (2000). AHG
%\noindent Chapter 5: Ohadi et al. (1999). AJHG.
%\noindent Chapter 7: Zhao J et al. (1999) AJHG
