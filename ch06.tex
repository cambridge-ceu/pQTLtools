% 2/8/2000 change ``candidate gene'' to ``modelling haplotypes''
% 15/10/2001 fix Xiong/Jin (2000) result and wrap everything within ``Method''

\chapter{Association in nuclear families}

\section{Introduction}

Typical linkage analysis assumes linkage equilibrium between disease locus and
markers.  Several recent works show that the incorporation of LD information is
necessary to increase power (e.g.  Terwilliger \& Ott 1994; Xiong \& Jin 2000).
This chapter presents some work on this subject for nuclear families, which is
relatively easy to study and generalisable to general pedigrees (e.g.  Morton
1955; Elston \& Steward 1971; Cannings et al.  1978; Bonney 1986; Martin et al.
2000; Abecasis, Cookson \& Cardon 2000).

Assumption of linkage equilibrium between disease locus and markers formed the
basis of allele frequency estimation of Boehnke (1991) and maximisation over
penetrances of Curtis \& Sham (1995).  Terwilliger \& Ott (1994) suggested
that lod score can be more appropriately obtained from the joint likelihood
linkage and association, treating haplotype frequencies as nuisance parameters.
However as recently pointed out by Xiong \& Jin (2000), they did not provide
algorithm incorporating LD parameters, nor did they use the correct critical
value for the statistics (Self \& Liang 1987) and conduct power analysis.
Family-based association methods assume a disease locus and a marker locus
being either in equilibrium or in disequilibrium.  There is a growing consensus
that a general likelihood framework combining linkage and association is needed
(Thomson 1995a, 1995b; Xiong \& Jin 2000; Slager et al.  2001).


\section*{Chapter aims}

This chapter attempts to use mathematical optimisation in combined linkage and
association analysis for nuclear families.  These include numerical experiments
similar to those in Terwilliger \& Ott (1994), Xiong \& Jin (2000).


\section{Methods}

Figure~\ref{ch23} is from Terwilliger \& Ott (1994), as an example of linkage
analysis allowing for LD.

\fig{4.5truein}{ch23.new}{Pedigree for association analysis between two marker loci}{ch23}

Initial haplotype frequencies were obtained from founder individuals using {\bf
EH}, and the input data, not given there, is as follows.
\begin{verbatim}
2 2
3 0 0
1 8 0
0 1 0
\end{verbatim}
Note that the first line indicates that there are two markers each with two
alleles, the following lines correspond to a $3\times 3$ table formed by the
three genotypes at each marker.

Denote the allele frequencies at marker 1 to be $p_1, p_2$ and those at marker
2 to be $q_1, q_2$.  A single disequilibrium parameter ($\delta$) is sufficient
to describe disequilibrium between two biallelic loci.  The variation between
$\delta=0$ and $\delta=-p_1q_1$ or $p_2q_2$ indicates no and complete linkage
disequilibrium.  This motivates us to incorporate this parameter into the
standard linkage analysis, so that the likelihood calculation is based directly
on haplotype frequencies and a proper optimisation procedure is used.

Consider a generalised single locus with allele frequencies $p$ and $q$ for the
normal and disease alleles, respectively. The penetrances of genotypes given
two, one and no disease alleles are $f_2$, $f_1$ and $f_0$.  Consider also a
marker locus with $n$ alleles, their associated haplotype frequencies being
$h_{11},...,h_{1n}, h_{21},...,h_{2n}$, conditional on individual's affection
status, the genotypic distribution is then obtained according to equation
(\ref{gendist}).  Alzheimer's model (Model 3 below) is used to define genotypic
probabilities, through which 22 models are generated (see table~\ref{models6})
by a SAS program given in Appendix B.

\begin{table}[h]
\caption{The 22 models compatible with Alzheimer's\label{models6}}
\centering
\vskip 0.3cm
\begin{tabular}{llllll}
\hline
model &   $q$  & $\delta$ &    $f_0$ &    $f_1$  &    $f_2$\\
\hline
  1   & 0.117  & 0.10027  & 0.049410 &  0.21760  &  0.95834\\
  2   & 0.130  & 0.10179  & 0.045761 &  0.20521  &  0.92021\\
  3   & 0.130  & 0.11310  & 0.050000 &  0.20000  &  0.80000\\
  4   & 0.143  & 0.10365  & 0.042493 &  0.19338  &  0.88008\\
  5   & 0.156  & 0.09454  & 0.034453 &  0.18342  &  0.97647\\
  6   & 0.156  & 0.10585  & 0.039598 &  0.18224  &  0.83867\\
  7   & 0.169  & 0.09707  & 0.031942 &  0.17163  &  0.92221\\
  8   & 0.182  & 0.09994  & 0.029809 &  0.16087  &  0.86821\\
  9   & 0.195  & 0.09184  & 0.022419 &  0.14611  &  0.95217\\
 10   & 0.195  & 0.10315  & 0.028020 &  0.15114  &  0.81520\\
 11   & 0.208  & 0.09539  & 0.021137 &  0.13677  &  0.88495\\
 12   & 0.221  & 0.08797  & 0.014451 &  0.11813  &  0.96561\\
 13   & 0.221  & 0.09928  & 0.020143 &  0.12860  &  0.82105\\
 14   & 0.234  & 0.09220  & 0.013983 &  0.11129  &  0.88580\\
 15   & 0.247  & 0.08545  & 0.008373 &  0.08962  &  0.95922\\
 16   & 0.247  & 0.09676  & 0.013721 &  0.10557  &  0.81225\\
 17   & 0.260  & 0.09035  & 0.008553 &  0.08621  &  0.86903\\
 18   & 0.273  & 0.08428  & 0.004232 &  0.06279  &  0.93176\\
 19   & 0.286  & 0.08986  & 0.004807 &  0.06335  &  0.83481\\
 20   & 0.299  & 0.08446  & 0.001783 &  0.03972  &  0.88456\\
 21   & 0.312  & 0.07940  & 0.000161 &  0.01230  &  0.93739\\
 22   & 0.325  & 0.08599  & 0.000580 &  0.02182  &  0.82145\\
\hline
\end{tabular}
\end{table}

For example with model 1, the haplotype frequencies are as follows \medskip

\begin{tabular}{llll}
\\
                &  \multicolumn{2}{c}{marker}\\ %\cline{2-3}
disease         &   allele 1 & allele 2 & total\\
disease allele  & 0.115479 &  0.014521 &   0.13\\
normal  allele  & 0.001521 &  0.868479 &   0.87\\
total           & 0.117    &  0.883    &   1\\
\\
\end{tabular}
\medskip

Only a subset of these models, here models 1, 3, 11 and 22 are used to
experiment on model identification.  The second marker is assumed to have
three, four, or five equifrequent alleles and in linkage equilibrium with the
first (allele frequency 0.1).  Nuclear families are generated from the {\bf
SIM} program (Chapter 5), with sibship size in accordance with truncated
Poisson distribution with parameter 3.  Families are kept only when there are
two or more family members are affected.  For the simulated data, numerical
optimisation is conducted to recover the simulated parameters.  If $q$ is
fixed, it is set to 0.13.  If $\delta$ is fixed, it is set to be the maximum
0.1311.  -2 log likelihood is minimised over penetrances in addition to
different combination of frequency and disequilibrium parameters.  The problem
is redefined as follows.

When $f_0=f_1$ and $f_2=1$, $f_1$ achieves the smallest value
${[K-q^2]}/{[1-q^2]}$, $f_1\ge \max({[K-q^2]}/{[1-q^2]},0)$, and when $f_0=0$
and $f_1=f_2$ the biggest ${K}/{[1-(1-q)^2]}$,
$f_1\le\min({K}/{[1-(1-q)^2]},1)$.

Given $f_1,q$, if $f_1<K$, $f_2$ achieves the smallest value when $f_0=f_1$,
i.e. ${[K-(1-q^2)f_1]}/{q^2}$; If $f_1>K$, the smallest $f_2$ is $f_1$, the
largest is when $f_0=0$, i.e.  $\min({[K-2q(1-q)f_1]}/{q^2},1)$.

Given $q, f_1, f_2$, $f_0={[K-2q(1-q)f_1-q^2f_2]}/{(1-q)^2}$.

Given the lower bound $L$ and upper bound $U$ of a parameter $x$, the problem
is then converted into a nonlinear optimisation with boundary constraints.
$x=L+c(U-L),$ where $c$ varies between 0 and 1.
\begin{eqnarray*}
 \min f_1&=&\max((K-q^2)/(1-q^2),0)\cr
 \max f_1&=&\min(K/(1-(1-q)^2),1)\cr
 f_1&=&\min f_1+c_1(\max f_1-\min f_1)\cr \cr
 \min f_2&=&\cases{f_1, & $f_1>K$ \cr  K-(1-q^2)f_1)/q^2,& otherwise}\cr\cr
 \max f_2&=&\min((K-2qpf_1)/q^2,1)\cr
 f_2&=&\min f_2+c_2(\max f_2-\min f_2)\cr
 f_0&=&(K-2qpf_1-q^2f_2)/p^2\cr
 \min\delta&=&\max(-qp_1,-pp_2)\cr
 \max\delta&=&\min(pp_1,qp_2)\cr
 \delta&=&\min\delta+c_3(\max\delta-\min\delta)\cr
 K_n&=&p^2f_0 + 2pqf_1 + q^2f_2
\end{eqnarray*}
Where $c_1-c_3$ are free parameters for the penetrances and disequilibrium
parameter, where $K_n$ is used to check for $K$.  Logistic transformation
$f(x)={1}/{[1+\exp(-x)]}$ removing the boundary constraints is an alternative
but not pursued.

The optimisation is performed with the conjugate gradient method of SAS by
repeatedly invoking {\bf LINKMAP}.  Statistical power can be appropriately
evaluated in terms of sample sizes based on the log-likelihood ratio
statistics.  For type I and type II error rates to be $\alpha=0.0001$,
$\beta=0.10$, the noncentrality parameter is 29.92 for $\chi^2$ degrees of
freedom 2.


\section{Results}

Based on Figure~\ref{ch23}, the likelihood ratio statistic of association
versus no association is 12.02, corresponding to a $p$ value of 0.0005.  The
estimated haplotype frequencies are (0.377219, 0.199704, 0.276627, 0.146450)
assuming independence and (0.576921, 0.000002, 0.076925, 0.346152) assuming
association; these yield a disequilibrium estimate of 0.199702.

These are used as initial values for {\bf ILINK} to maximise the likelihood
over haplotype frequencies using the whole pedigree to establish the phase.
When the markers are unlinked the estimate (0.576837, 0.000000, 0.076995,
0.346169) is very similar to those from {\bf EH}.  By allowing for linkage
between the two markers, and estimate the recombination and haplotype
frequencies jointly the results become (0.315367, 0.257574, 0.341849, 0.085210)
and disequilibrium value of -0.061852.  The values of -2 log-likelihood from
{\bf ILINK} are 92.9909 and 97.9948120 so the optimisation process is not
satisfactory.  Such a seemingly simple example shows that the likelihood
surface is not simple.

The same result occurs from SAS optimisation procedure:  fixing recombination
rate $\theta=0.5$ and starting with equal haplotype frequencies, -2
log-likelihood value drops from 108.13096 to 92.991507 with estimates
(0.5759099, 0, 0.0782231 and 0.345867) under Sun, -2 log-likelihood 85.335481
and haplotype estimates (0.6622288, 0.0000000, 0.1902562, 0.1475151) under DEC
Alpha, while allowing for linkage, -2 log-likelihood remains to be 97.975395
with haplotype frequency estimates (0.3168155, 0.2601083, 0.3370297, 0.0860466)
under DEC Alpha, suggesting a very flat surface with respect to $\theta$.  It
seems the failure of SAS may be due to convergence to a local minimum, but log
likelihood remains unchanged with moderate change in parameter value.

The result is not examined in detail here for the simulated data from {\bf
SIM} due to some vexing problems.  First, the scheme from which the sample is
simulated may lead to ascertainment bias.  Second, it discards families
potential for LD, such as family trios (two parents and one child), which is
typical data for TDT designs for test of linkage and association.  Finally, the
implementation in {\bf LINKAGE} is quite limited.


\section{Discussion}

Most current procedures have their limitations.  For example, {\bf ILINK} can
provide maximum likelihood estimation of allele frequencies but the penetrances
are assumed known, and most problems are difficult for the GEMINI routine when
highly polymorphic markers are involved.  In fact these could be special cases
of family likelihood which takes into account allele frequencies, penetrances,
disequilibrium, in a combined linkage and association framework.  In priciple
we can use any linkage analysis routine as function calculator for
high-performance optimisation routine, although when the likelihood computation
itself is computer intensive, especially for extended pedigrees with many
loops.

The likelihood in a typical linkage analysis is affected by founder allele
frequencies.  It takes little reflection to recognise that founder LD has
similar role, and that even families with one affected child can provide useful
information about linkage disequilibrium.  An explicit likelihood framework is
desirable to explore the potential increase in power, although in reality it
remains appealing to continue experiments on programs such as {\bf LINKAGE},
with its capability to handle haplotype frequencies.

For single markers it is possible to obtain analytical power via method
described in Thomson (1995a, 1995b) or Xiong \& Jin (2000).  A nuclear family
with three siblings (called 3S below in their convention) was examined in
Figure 1 of Xiong \& Jin (2000).  Specifically the family had two unaffected
parents and three affected siblings, all were genotyped at a biallelic marker
($p_1$=frequency of allele 1, $p_2$=frequency of allele 2) and both parents and
the first two siblings were 1/2 heterozygous, while the third sibling is 1/1
homozygous.  Now under Mendelian recessive model ($p_d$=frequency of normal
allele, $p_D$=frequency of disease allele), the likelihood is
$4(p_dp_Dp_1p_2)^2[\theta^2(1-\theta)^4+\theta^5(1-\theta)+(1-\theta)^5\theta+
\theta^4(1-\theta)^2]$, with $\theta$ being the recombination rate between
disease and marker loci.  The lod score curve is shown in Figure~\ref{3slod}.

%\ffig{4.5truein}{ch23.ps}{ch23}
\begin{figure}
\centerline{\epsfig{figure=lod.ps,width=4.2in,height=3in}}
\caption{LOD score of 3S family assuming linkage equilibrium\label{3slod}}
\end{figure}

Assume $p_1=0.1$, $p_D$=0.01, this function is approximately maximised at
$\theta=0.2$ with lod score 0.12.  By allowing for LD between disease and
marker loci ($\delta$), the lod score function can be expressed as $4(c_1
c_4)^2 (1-\theta)^4 \theta^2+4c_1 c_2 c_3 c_4 [(1-\theta)^5 \theta+\theta^5
(1-\theta)]+4(c_2 c_3)^2 \theta^4 (1-\theta)^2$, where $c_1=p_1 p_D+\delta$,
$c_2=p_1 p_d-\delta$, $c_3=p_2 p_D-\delta$, $c_4=p_2 p_d+\delta$.  Apparently
the lod score 1.06 in their table I would be 0.92.  It is rather unfortunate
that even in such a simple case it would be difficult with numerical
optimisation.  We can evaluate the lod score function using fine grid points of
recombination rate and disequibrium value.  Many points exceeded their optimal
solution (e.g.  1.554 when $\theta=0.325$, $\delta=0.009$).
Figure~\ref{3slod3d} shows the lod score surface using GNUPlot.

\begin{figure}
\centerline{\epsfig{figure=lod3d.ps,width=4.2in,height=3in}}
\caption{LOD score of 3S family assuming linkage disequilibrium\label{3slod3d}}
\end{figure}

The lod score function indeed favors larger disequilibrium value.  In general,
let $S_{ijkl}$ be the probability of parental genotype $(i,j)\times (k,l)$) at
a multiallelic marker, $S_{ijkl}(\theta, 0)=p_ip_jp_kp_l$ under linkage
equilibrium; $S_{ijkl}(\theta,\delta)=p_ip_jp_kp_lW(\theta,\delta)$ under
linkage disequilibrium ($W(\theta,\delta)$ is a function of disease/marker
allele frequencies and disequilibrium values between them).  The noncentrality
parameter for family trios (two parents and an affected sibling, called S
families) is $$\lambda_S(\theta,\delta)=4N
\sum_{ijkl}p_ip_jp_kp_lW(\theta,\delta)\ln W(\theta,\delta)$$ For single
mutation, and from $\delta(t)=\delta_0e^{\theta t}$, with $t$ being the number
of generations that a single mutation was introduced, they obtained the
approximation $$\lambda_S(\theta,\delta) \approx
\frac{4N[\theta^2+(1-\theta)^2]}{p^2(A)}\left[ (f_{DD}-f_{Dd})p_D
+(f_{Dd}-f_{dd})p_d\right]\frac{1-p_1}{p_1} e^{-2\theta t}$$ Where $p$'s and
$p_D$ are the marker and disease allele frequencies.
$P(A)=\sum_{r}\sum_{t}f_{rt}p_rp_t$, $r,t=d,D$.  A QBASIC program for this
calculation is given in Appendix B.  It was surprising that all the 22 models
virtually yielded the same likelihood and noncentrality parameter, which needs
to be further investigated.

Lake et al.  (2000) also proposed tests of association in the presence of
linkage.  Under the null hypothesis there is linkage but no association,
whereas under the alternaive there are both linkage and association.  They also
proposed an empirical variance-covariance estimator that is robust to sibling
marker-genotype correlation.  This is rather appealing but only appropriate for
single markers.  Recently, Slager et al.  (2001), Huang \& Jiang (2001)
proposed another framework of combined linkage and association analysis.  Their
method used phase probability as indication of association, for it can be shown
that equal phase probabilities for repulsive phases in the usual linkage
analysis is true only when there is linkage equilibrium.  They expressed the
likelihood as a function of both recombination and phase probabilities.  Again
the threshold for significance is determined by mixture of $\chi^2$
distributions (Self \& Liang 1987).  They showed that for cases involving two
or more affected siblings under a variety of scenarios their method is more
powerful than both TDT and affected sibpair mean test.

Mathematical optimisation depends both on the problem and the numerical
techniques (e.g.  Morton et al.  1983).  A successful example is {\bf Mx}
(http://views.vcu.edu/mx/), a freely available program for structural equation
modelling and widely used for twin data analysis.  Both {\bf Mx} and {\bf PAP}
use {\bf NPSOL} (http://www.sbsi-sol-optimize.com/NPSOL.htm).  Others include
{\bf CFSQP} (http://gachinese.com/aemdesign/fsqpcontents.htm) and {\bf SEARCH}
(http://www.biomath.medsch.ucla.edu/faculty/klange/software.html).  To make the
models feasible these tools are indispensible.


\section{Bibliographic notes}

The method of Boehnke (1991) is now part of {\bf MENDEL}.  There is renewed
interest in combined linkage and association analysis (Terwilliger et al.
1992; Terwilliger \& Ott 1994; Keavney et al.  1998; Terwilliger \& G\"{o}ring
2000; Xiong \& Jin 2000).

{\bf GENEHUNTER} obtains most probably haplotype frequency estimate via Viterbi
algorithm for hidden Markov model.  {\bf SimWalk2} (Sobel \& Lange 1996) uses
simulation-based method to extract haplotype information.  {\bf PedManager}
(http://waldo.wi.mit.edu /ftp/distribution/software/), {\bf Maga2}
(Mukhopadhyay et al.  1999) can also be used to estimate allele frequencies
based on family data, but they do not seem to be based on haplotype analysis.
It is likely that programs designed for family data could induce bias owing to
ascertainment.

Self \& Liang (1987) gave asymptotic distribution of log-likelihood ratio
statistic under several nonstandard scenarios.  Related work on the subject
also includes Chernoff (1954), Davis (1977, 1987), Geyer (1994).  Genetic
examples include Holmans (1993), Chernoff \& Lander (1995), Liang \& Rathouz
(1999).
