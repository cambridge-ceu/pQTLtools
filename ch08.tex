% create Friday 15-12-00, last modified 21-7-01

\chapter{Bells and Whistles}

This chapter gives a short summary of the major findings and discusses possible
extensions.

\section{Major findings}

Several scenarios of statistical power analysis in human genetic linkage and
association have been investigated.  The use of computer simulation methods and
power of TDT and other association designs are also explored.  It is found that
assuming full marker informativeness, parametric and nonparametric linkage
statistics have almost equal power, but nonparametric linkage statistics are
anticonservative in some families.  In case-control association analysis with
unrelated individuals, a heterogeneity statistic has comparable power to
parametric tests and comparable to the ordinary likelihood ratio test for
contingency table.  Depending on mode of inheritance, the classic case-control
and family designs are equally important.  With availability of large number of
SNPs, the role of other types of markers in detecting LD can not be overlooked.
The power of different designs may be variable depending on the underlying
disease model.

The investigation of statistical power has important implications.  It should
be seen as part of a dynamic process in line with methodological development.
Indeed genetic analysis has been one of the most fascinating and motivating
fields for modern probabilistic and statistical theory, and an eminent recent
feature is that the sheer amount of information will surpass any sophisticated
mathematical model and computing resource available.  As for human genetic
linkage and association analysis, the large number of genetic markers in
junction with the complexity of large family poses special challenges, and
methods for combined linkage and association analysis for multiplex families
and other data sources such as unrelated individuals are yet to be fully
integrated.


\section{Possible extensions}

While many other aspects such as QTL analysis and DNA pooling are left out in
this thesis, possible extensions, by chapter, are sketched as follows.

\begin{itemize}

\item{\em Linkage tests for arbitrary pedigrees.} For large pedigrees,
complete enumeration and full marker informativeness will soon become unwieldy,
so simulation-based methods must be adopted.  Other factors such as marker
allele frequencies will have to be considered.

\item{\em Case-control analysis.} The major problem is the large number of
parameters when multiple and/or multiallelic markers are involved, so
heuristic methods and parsimonious models have to be used.

\item{\em Multilocus model for fine mapping.} The deterministic model can
to be strengthened with more loci.  It would be more desirable to take into
more account of human evolutionary history, i.e., by stochastic simulations
and coalescent models.

\item{\em Conditional simulation for homozygosity mapping.} While there are
programs for MCMC linkage analysis, programs for routine power analysis
remain to be developed.

\item{\em LD mapping using families.} While this thesis only reveals some
issues of combined linkage and association, most computer programs are not
ready for practical data analysis.  Good optimisation techniques and programs
are necessary but not properly incorporated. Alternative models could also be
investigated. It is desirable to obtain the power table for the generated
models along the line of Xiong \& Jin (2000).

\item{\em Comprehensive power analysis of TDT method.} This should apply to
multiplex pedigrees with multiple markers, and possibly with other
factors such as incomplete information, missing data, covariates and human
evolutionary history.

\end{itemize}

Complex traits need a good working model to distinguish the contributing
factors such as small sample size, insufficient markers, multiple testing,
nonuniform linkage disequilibrium across genome, genotype error, and phenotype
definition (Kruglyak \& Lander 1995, Risch 2000; G\"{o}ring \& Terwilliger
2000; Cardon \& Bell 2001), heterogeneity, gene-environmental interactions,
population history not fully used, etc.  The demographic and evolutionary
history of the current world population, while important for applications such
as admixture mapping (e.g.  Laan \& P\"{a}\"{a}bo 1997; Chakraborty \& Weiss
1988), remains uncertain, and genetic distance does not necessarily mean
physical distance (Morton 2000).  LD is not merely dependent upon physical
distance, but also on differing allele frequencies and age of mutation.
High-throughput genome screening may need stringent prior probability for
linkage and association (Risch 2000).

It has been several years after Lander \& Schork (1994) reviewed then the
research paradigms in genetic analysis of complex traits.  As the Human Genome
Project is approaching its targets these have gradually shifted (e.g., Risch \&
Merikangas 1996; Schork et al.  1998; Guo \& Lange 2000; Johnson \& Todd 2000).
This change is attributable to the interplay of both theory and application.  A
classic example was Spielman et al.  (1993) in re-analysing of their
insulin-dependent diabetes mellitus and a 5' flanking polymorphism of the
insulin locus, which also initiated interests on TDT.  Consortium work such as
GAW has been motivated to develop and evaluate various statistical techniques.
Methodological advance requires study designs that combine recent developments
in statistics, biology and epidemiology.


\section{Epilogue}

This thesis is largely done as a result of ongoing papers I have been involved
as a PostDoc research worker in several Wellcome project grants.  The
consequence is two-fold.  On the one hand, it does not fully reflect my work
such as haplotype analysis; on the other, given the dynamic nature of the field
and my own limitation, it would have been a dream to provide a comprehensive
account of all topics in the thesis, and it is more appropriate to see it as a
timely summary.  Initially trained in medicine and medical statistics, it is a
result of many years of {\em wading into the swamp} of literature of
statistical genetics, mathematical models, algorithms, data structure and
computer programming.  The joy of being involved and presenting work in this
exciting area is beyond words.
